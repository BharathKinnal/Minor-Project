\chapter{Introduction}

In present day technology human-machine interaction is growing in demand and machine needs to understand human gestures and emotions. If machine can identify human emotion, it can understand human behavior better, thus improving the task efficiency. It can serve as a vital measurement tool for behavioral science, socially intelligent software can be developed which can be used for robots. Emotions are the strong feelings which are governed by the surroundings and play a great role in daily task like decision making, learning, attention, motivation, coping, perception, planning, cognition, reasoning and many more, which leads to emotion recognition a big research field. Emotion recognition can be done by text, vocal, verbal and facial expression. Facial expression analysis is one of the most important components for emotion recognition. Facial emotion recognition from 2D images is well studied field but lack of real-time method that estimates features even low quality images. Most of the work are based on frontal view images of the faces.\par
The method proposed is a real-time emotion recognition system that recognizes basic emotions like anger, disgust, happiness, surprise and neutral using CMU MultiPIE database consisting 2D images with different illumination and poses. The software system developed using our proposed method is deployed on Raspberry Pi II as it can be used with robots as the size of Raspberry Pi II is very small, lightweight and very less power supply is needed for it. As a result it can be mounted over any robot very easily and can be used for many applications such as surveillance security, monitoring senior citizen or children at home, monitoring critical patients in ICU, for customer satisfaction and many more.

\section{Motivation}

If machines can identify human emotions, it can understand human behaviour better which   improves task efficiency. Emotions greatly affect decision making, learning, attention, motivation, coping, planning, cognition, reasoning etc. Facial emotion detection from 2D images is well studied field but lack of real-time method to estimate features gives the necessity to develop a method for the same Parallelization of the real-time process, which might lead to higher speeds keeping accuracy at it peak.